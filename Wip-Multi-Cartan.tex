%+--------------------------------------------------------------------------------------+
%| CHEATSHEET: Multivectors version of Cartan calculus formulas
%| Author: Antonio miti
%| 
%| 
%+--------------------------------------------------------------------------------------+


\documentclass[a4paper,12pt]{scrartcl}

%RWZ
\newcommand{\g}{\ensuremath{\mathfrak{g}}}

%CFRZ
\newcommand{\LX}{\mathfrak{X}^{\wedge {\bullet}}}
\renewcommand{\L}{\mathcal{L}}
\renewcommand{\deg}[1]{\left \lvert #1 \right \rvert}
\newcommand{\ExtD}{\textrm{d}}
\newcommand{\Lie}{\mathcal{L}}

%
\usepackage{pdflscape}
\usepackage{tabularx}
\usepackage{array}
\setlength{\extrarowheight}{6mm}
\usepackage{geometry}
 \geometry{
 a4paper,
 total={185mm,277mm},
 left=5mm,
 right=5mm,
 top=5mm,
 bottom=5mm,
 }
 
\usepackage[subpreambles=true]{standalone}
 
\usepackage{amssymb}
\usepackage[leqno]{amsmath}
\usepackage{amsfonts}
\usepackage{mathtools}
\usepackage{hyperref}

\usepackage[symbol]{footmisc}
\renewcommand*{\thefootnote}{\fnsymbol{footnote}}

	\renewcommand{\d}{\textrm{d}}
	\providecommand{\codiff}{\delta}
	\providecommand{\laplacian}{\Delta}
	\providecommand{\Lie}{\mathcal{L}}%\pounds


%https://tex.stackexchange.com/questions/48980/whole-page-table-with-tabularx

\begin{document}
  \begin{landscape}
    \thispagestyle{empty}
    \noindent
    \paragraph{MULTIVECTORS CARTAN CALCULUS (WIP)}
    	\mbox{}\\
        $\quad$Suppose $M$ is a smooth manifold, $x^\mu$ a coordinate chart. Denote by $\Omega(M)$ the algebra of differential forms on $M$ and by $\mathfrak{X}(M)$ the $C^\infty(M)$-module of vector fields.  \\

  \end{landscape}
  
  \section{Sketchy part}
  	Suggested by arXiv:1610.05592v1,  arXiv:1304.2051
	\begin{itemize}
		\item
			Let $M$ be a manifold and let $\Omega$ be a not necessarily closed differential form on $M$. For all $m \geq 1$ and all vector fields $v_1,\dots,v_m$ in the Lie algebra $\mathfrak X(M)$ we have:
			\begin{align*} 
			(-1)^{m}d \iota(v_{1} \wedge\cdots \wedge v_{m}) \Omega &= 
			\iota(\partial(v_{1}\wedge\ldots\wedge v_{m}))\Omega
			+\sum_{1=1}^{m} (-1)^{i} \iota( v_{1} \wedge \cdots
			 \wedge \hat{v}_{i} \wedge \cdots \wedge {v}_{m})\mathcal{L}_{v_i}\Omega\\
			&+ \iota( v_{1} \wedge \cdots
			 \wedge {v}_{m}) d\Omega.
			\end{align*}	
		\item
			Given a differential form $\Omega\in \Omega^\bullet(M)$ and a multi-vector field $Y\in \Gamma(\Lambda^m TM)$, the \emph{Lie derivative of $\Omega$ along $Y$} is defined as a graded commutator,
			by $\mathcal{L}_Y\Omega:=d\iota_Y\Omega-(-1)^m\iota_Yd\Omega$.	
		\item
			This definition allows to combine the first and last term in the above formula into a Lie derivative.
			Hence the above formula can be written 
			$\mathcal{L}_{v_{1} \wedge \cdots
			\wedge {v}_{m}}\Omega 
			=(-1)^m[
			\iota(\partial(v_{1}\wedge\ldots\wedge v_{m}))\Omega+\sum_{1=1}^{m} (-1)^{i} \iota( v_{1} \wedge \cdots
			 \wedge \hat{v}_{i} \wedge \cdots \wedge {v}_{m})\mathcal{L}_{v_i}\Omega]$.	
		
		\item
			$[y, p \wedge q ] = ?$
		\item		
			Let $p\in \Lambda^k\g$ and $q\in\Lambda^l\g$. Then
			$$\partial(p\wedge q)=\partial(p)\wedge q+ (-1)^k p\wedge \partial(q)+(-1)^k[p,q],$$
			where $[x_1\wedge...\wedge x_k,y_1\wedge ...\wedge y_l]=\sum (-1)^{i+j}[x_i,y_j]\wedge x_1 \wedge ...\wedge \hat x_i\wedge ...\wedge x_k \wedge y_1 \wedge ...\wedge \hat y_j\wedge ...\wedge y_l$. 
			
			Proof:\\
			It is sufficient to prove the assertion for monomials
			$p= x_1\wedge\ldots\wedge x_k$ and $q=x_{k+1}\wedge\ldots\wedge x_{k+l}$.
			In that case $\partial(p\wedge q)$ is given by a 
			sum over indices $i,j$ with ${1 \leq i < j \leq k+l}$. Splitting it into sums over ${ i < j \leq k}$, ${k< i < j }$ and $ i \leq k < j$ proves the assertion.
		\item
			The bracket $[\cdot,\cdot]:\Lambda^\bullet\mathfrak g\times \Lambda^\bullet\mathfrak g\to \Lambda^\bullet\mathfrak g$ defined above turns $\Lambda^\bullet\mathfrak g$ into a Gerstenhaber algebra.

		\item
			$$\iota_{a^{(k)}} \dot \iota_{b^{(p)}} = (-)^{k p} \iota_{b^{(p)}} \dot \iota_{a^{(k)}} $$
			
		\item
			Multi-vector fields bracket ( Schouten–Nijenhuis bracket)
			$$
				[v_1 \wedge \cdots \wedge v_m , w_1 \wedge \cdots \wedge w_n] = 
				\sum_{i,j}(-1)^{i+j}[v_i,w_j] \wedge v_1 \wedge \cdots \wedge \hat{v_{i}} \wedge \cdots \wedge v_m \wedge w_1 \wedge \cdots \wedge \hat{w_{j}} \wedge \cdots \wedge w_n
			$$

			$$			
			[ f , v_1 \wedge \cdots \wedge v_m] = -\iota_{df}(v_1 \wedge \cdots v_m)
			$$
			
			$(\mathfrak{X}^\bullet (M) , [\cdot,\cdot] )$ \href{https://en.wikipedia.org/wiki/Gerstenhaber_algebra}{Gerstenhaber algebra}.
		
		\item
			take $v \in \mathfrak{X}, p \in \mathfrak{X}^k$ then
			$$
				\mathfrak{L}_v \iota_p \omega = \iota_{[v,p]} \omega + \iota_p \mathfrak{L}_v \omega
			$$
		
		\item
			Per fare schema di multicartan servono regole di commutazione con $p \in \mathfrak{X}^\bullet$:
			$$ [ p,p'], \mathfrak{L}_p, \iota_p, d$$
			
		\item
			$\partial \equiv \partial_k:  \Lambda^{k} {\mathfrak g} \to \Lambda^{k-1} {\mathfrak g}$  via
$$
\partial (\xi_1 \wedge \xi_2 \wedge \dots \wedge \xi_k) := \sum_{1\leq i< j \leq k} (-1)^{i+j}\, [\xi_i, \xi_j] \wedge \xi_1 \wedge \dots {\hat \xi}_i \wedge \dots \wedge {\hat \xi}_j \wedge \dots \xi_k
$$
(with $\hat{}$ denoting deletion as usual and with $\partial_0 = 0$; one has $\partial^2 = 0$). \par
		\item
		The \textbf{Schouten bracket}
% $[\cdot,\cdot] \maps \LX(M) \times \LX(M) \to \LX(M)$ 
% is a degree $-1$ Lie bracket which satisfies the graded Leibniz rule with respect to the
% wedge product.
of two decomposable multivector fields
$u_{1} \wedge \cdots \wedge u_{m}, v_{1} \wedge \cdots \wedge v_{n}
\in \LX(M)$ is
\begin{multline} \label{Schouten}
\left [ u_{1} \wedge \cdots \wedge u_{m}, v_{1} \wedge \cdots \wedge
  v_{n} \right] 
= \\\sum_{i=1}^{m} \sum_{j=1}^{n} (-1)^{i+j} [u_{i},v_{j}]
\wedge u_{1} \wedge \cdots \wedge \hat{u}_{i} \wedge  \cdots \wedge
u_{m}\\
\quad \wedge v_{1} \wedge \cdots \wedge \hat{v}_{j} \wedge \cdots \wedge v_{n},
\end{multline}
where $[u_{i},v_{j}]$ is the usual Lie bracket of vector fields.

The \textbf{interior product} of a decomposable
multivector field $v_{1} \wedge \cdots \wedge v_{n}$ with $\alpha \in \Omega^{\bullet}(M)$ is
\begin{equation} \label{interior}
\iota(v_{1} \wedge \cdots \wedge v_{n}) \alpha = \iota_{v_{n}} \cdots
\iota_{v_{1}} \alpha,
\end{equation}
where $\iota_{v_{i}} \alpha$ is the usual interior product of vector
fields and differential forms. 
% The interior product of an arbitrary
% multivector field is obtained by extending by $\cinf(M)$-linearity. 

The \textbf{Lie derivative} $\L_{v}$ of a differential form along a multivector field $v \in
\LX(M)$ is the graded commutator of $d$ and $\iota(v)$:
\begin{equation} \label{Lie}
\L_{v} \alpha =  d \iota(v) \alpha - (-1)^{\deg{v}} \iota(v) d\alpha,
\end{equation}
where $\iota(v)$ is considered as a degree $-\deg{v}$ operator.

The last identity we will need is for the graded commutator of
the Lie derivative and the interior product. Given $u,v \in
\LX(M)$, it follows from  \cite[Proposition A3]{Forger} that
\begin{equation} \label{commutator}
\iota([u,v]) \alpha = (-1)^{(\deg{u}-1)\deg{v}} \L_{u} \iota(v)  \alpha - \iota(v)\L_{u} \alpha.
\end{equation}
	
	\item Prove that the following relations holds when $X,Y \in \mathfrak{X}^k(M)$
				\begin{align}
					\ExtD^2 &= 0 \label{cartfirst}\\
					\ExtD \Lie_X - \Lie_X d &= 0 \\
					\ExtD \iota_X + \iota_X d &= \Lie_X \label{magic}\\
					\Lie_X \Lie_Y - \Lie_Y \Lie_X &= \Lie_{[X,Y]} \\
					\Lie_X \iota_Y - \iota_Y \Lie_X &= \iota_{[X,Y]}\\
					\iota_X \iota_Y + \iota_Y \iota_X &= 0 \label{cartlast}
				\end{align}	
	


	\item Another useful formula
				Controlla se è valida solo per n-forme!
		\begin{displaymath}
			\ExtD \omega (X_0, \ldots, X_n) = 
			\sum_i (-)^i X_i\left( \omega(X_0,\ldots,\hat{X_1},\ldots,X_n \right) +
			\sum_{i < j} (-)^{(i+j)} \omega\left([X_i,X_j],X_0,\ldots,\hat{X_i},\ldots,\hat{X_j},\ldots, X_n \right)
		\end{displaymath}
		or
		\begin{displaymath}
		\iota_{X_0\wedge \ldots X_n} \ExtD \omega =
		\sum_i (-)^i \Lie_{X_i} \iota_{X_0\wedge\ldots\wedge \hat{X_i}\wedge\ldots\wedge X_n} \omega
		+ \iota_{\partial X_0 \wedge\ldots\wedge X_n} \omega		
		\end{displaymath}


	\end{itemize}








\end{document}
